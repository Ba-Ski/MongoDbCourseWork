\documentclass[a4paper]{article}
\title{Database report}
\author{baski}
\date{june 2016}
\usepackage{indentfirst}
\usepackage[warn]{mathtext}
\usepackage{subcaption}
\usepackage{float}
\usepackage{amsmath}
\usepackage{graphicx}
\usepackage{enumitem}
\usepackage[T2A]{fontenc}
\usepackage[utf8]{inputenc}
\usepackage[russian]{babel}
\usepackage{listings,a4wide,longtable,amsmath,amsfonts,graphicx,tikz,tabularx}
\usepackage[final]{pdfpages}
\usepackage{color}
\usepackage{verbatim}

\definecolor{myblue}{RGB}{59,108,157}


\lstset{
	basicstyle=\footnotesize,
	breakatwhitespace=false,
	breaklines=true,
	extendedchars=true,
	commentstyle=\color{myblue},
	keepspaces=true,
	keywordstyle=\bfseries,
  language=[Sharp]C,
	numbersep=5pt,
	numberstyle=\tiny,
	showspaces=false,
	showstringspaces=false,
	showtabs=false,
	stepnumber=1,
	stringstyle=\emph,
	tabsize=8
}

\begin{document}
\begin{titlepage}
	\centering
	{\scshape\LARGE Университет ИТМО\par}
	\vspace{0.3cm}
	{\scshape\Large Кафедра вычислительной техники\\
	\vspace{0.2cm}
	<<Системы баз данных>>\par}
	\vspace{4cm}
	{\scshape\large Курсовая работа\\
	\vspace{0.2cm}
	на тему:\par}
	\vspace{0.2cm}
	{\itshape\large,,MongoDB: Веб-блог''\/}
	\vfill
	{\large
	{\raggedright{}
		\hspace{8.5cm}Выполнили: \\
		\hspace{8.5cm}студенты гр. P3315\\
		\hspace{8.5cm}Бонковски Патрик\\
		\hspace{8.5cm}Авраменко Илья\\
		\hspace{8.5cm}Преподаватель: \\
		\hspace{8.5cm}Беликов П.А.\\
	\par}
	\vfill
	Санкт-Петербург\\
	2016\\}
\end{titlepage}
\section{Описание прикладной области} Веб-блог - достаточно популярное явление в современном мире. Веб-блог -- это интернет сайт регулярно добавляемые записи, содержащие текст, изображения или мультимедиа. Блоги -- обычно публичны, то есть предполагают сторонних читателей, а также в своем большинстве обеспечивают возможность читателям оставлять свои комментарии по поводу прочитанной статьи в блоге. Информация публикуемая в блогах может быть посвященная только одной теме или же затрагивать
различные темы, но связанные одной областью. Хорошим примером является коллективный блог ``Хабрахабр''. Общая тематика блога -- сфера IT, в которой статьи группируются уже в узкоспециализированные темы, например: Linux, web-разработка, C++, Алгоритмы, итд, что позволяет удобно осуществлять поиск по по интересующим пользователям темам. \par
Любой блог, как и любой сайт, который должен хранить достаточно большие объемы информации и осуществлять к ним эффективные операции поиска, вставки и модификации, должен в этих целях взаимодействовать с базой данных. В нашем случае выбрана конкретная база данных, а именно документоориентированная система управления базами данных MongoDB. Благодаря особенностям способа хранения данных, MongoDB должна обеспечить быстрое предоставление конечному пользователю запрашиваемые данные.
\section{Модель базы данных}
База данных будет состоять из трех основных сущностей:
\begin{itemize}[noitemsep]
  \item статья,
  \item пользователь,
  \item комментарий.
\end{itemize}
В связи с тем, что mongodb -- документоориентированная база данных, а чтение одного документа является атомарной и очень быстрой операцией, то имеет смысл сущность комментарий сделать частью сущности статья, то есть сделать комментарий встроенным документом. В результате в базе данных будут тсодержаться только две коллекции документов:
\begin{enumerate}[noitemsep]
  \item статьи,
  \item пользователи
\end{enumerate}
Документ из коллекции статьи (posts)  содержит следующие поля:
\begin{enumerate}[noitemsep]
  \item id,
  \item author,
  \item title,
  \item content,
  \item date,
  \item tags,
  \item comments
    \begin{enumerate}
      \item author,
      \item date,
      \item content,
      \item nesting level.
    \end{enumerate}
  \item url.
\end{enumerate}
\begin{verbatim}
{                                                                    
        "_id" : ObjectId("5760c34449a4ffde0a1bb189"),                
        "title" : "awesome title 20676",                             
        "author" : "cooolGuy1699967",                                
        "content" : "boring article",                                
        "comments" : [                                               
                {                                                    
                        "author" : "cooolGuy1964430",                
                        "content" : "boring content",                
                        "date" : ISODate("1950-01-20T21:00:00Z"),    
                        "nestingLevel" : 1                           
                },                                                   
                {                                                    
                        "author" : "cooolGuy1312513",                
                        "content" : "boring content",                
                        "date" : ISODate("1950-04-11T21:00:00Z"),    
                        "nestingLevel" : 1                           
                },                                                   
                {                                                    
                        "author" : "cooolGuy881235",                 
                        "content" : "boring content",                
                        "date" : ISODate("1950-05-17T21:00:00Z"),    
                        "nestingLevel" : 1                           
                },                                                   
                {                                                    
                        "author" : "cooolGuy382200",                 
                        "content" : "boring content",                
                        "date" : ISODate("1949-08-23T21:00:00Z"),    
                        "nestingLevel" : 1                           
                }                                                    
        ],                                                           
        "tags" : [                                                   
                "Amazon Web Services",                               
                "Amazon Web Services",                               
                "Amazon Web Services",                               
                "*nix"                                               
        ],                                                           
        "date" : ISODate("1949-06-23T21:00:00Z"),                    
        "url" : "boring url"                                         
}
\end{verbatim}
Документ из коллекции пользователи (users):
\begin{enumerate}[noitemsep]
  \item id,
  \item nick,
  \item name,
  \item registration date,
  \item from.
\end{enumerate}
\begin{verbatim}
{
        "_id" : ObjectId("5745809b0f1f3c209c600601"),
        "nick" : "HoochieMen",
        "registerDate" : ISODate("2007-09-26T14:57:00Z"),
        "name" : "Алексей Мелихов",
        "from" : [
                "Россия",
                "Москва и Московская обл.",
                "Москва"
        ]
}
\end{verbatim}

Связь между документами осуществляется по полю author из статьи и nick из пользователя.
\section{Генерация данных}

Основные данные были непосредственно считаны из Хабрхабра, путем извлечения информации из html кода страниц сайта. Считыванием занималась программа написанная на С# и использовавшая библиотеку Html Agility Pack. В связи с тем, что извлечения информации таким способом происходило довольно медленно, то для достижения наличия миллионов записей в нашей базе данных часть данных была сгенерирована псевдослучайным образом с помощью следующего JavaScript скрипта:
\begin{verbatim}
var places = [
    "St-Petersburg", "Moscow", "Rostov", "Vladyvostok", "Kazan",
    "Minsl", "Brest", "Warsaw", "Yekaterinburg", "Sochi",
    "New York", "Berlin", "London", "Smolensk", "Habarovsk",
    "Stockholm", "Vladimir", "Kiev", "Paris", "Pushkin", "Omsk",
    "Tomsk", "Novosibirsk", "Krasnodar", "Volgograd", "Saratov",
    "Samara", "Tyumen", "Tver", "Simferopol", "Sevastopol", "Krasnyarsk",
    "Murmansk", "Astrakhan"
]

var firstName = ["Runny", "Buttercup", "Dinky", "Stinky", "Crusty",
    "Greasy", "Gidget", "Cheesypoof", "Lumpy", "Wacky", "Tiny", "Flunky",
    "Fluffy", "Zippy", "Doofus", "Gobsmacked", "Slimy", "Grimy", "Salamander",
    "Oily", "Burrito", "Bumpy", "Loopy", "Snotty", "Irving", "Egbert", "James",
    "Waffer", "Lilly", "Rugrat", "Sand", "Fuzzy", "Kitty",
    "Puppy", "Snuggles", "Rubber", "Stinky", "Lulu", "Lala", "Sparkle", "Glitter",
    "Silver", "Golden", "Rainbow", "Cloud", "Rain", "Stormy", "Wink", "Sugar",
    "Twinkle", "Star", "Halo", "Angel"];

var lastName = ["Snicker", "Buffalo", "Gross", "Bubble", "Sheep",
    "Corset", "Toilet", "Lizard", "Waffle", "Kumquat", "Burger", "Chimp", "Liver",
    "Gorilla", "Rhino", "Emu", "Pizza", "Toad", "Gerbil", "Pickle", "Tofu",
    "Chicken", "Potato", "Hamster", "Lemur", "Vermin", "Smith", "Brown", "Ihateusa",
    "Obama", "Gay", "Kfc", "Nigga", "Chef", "Cartman"];

function randomDate(start, end) {
    return new Date(start.getTime() + Math.random() * (end.getTime() - start.getTime()));
}

function randomInt(min, max) {
    return Math.floor(Math.random() * (max - min + 1)) + min;
}

function randomNickFormDB(count) {
    var randDate = randomDate(new Date(1905, 1, 1), new Date());
    var res = db.users.findOne({ registerDate: { $gt: randDate } });
    if (res == null)
        res = db.users.findOne({ registerDate: { $lt: randDate } });
    return res.nick;
}
function randomTagFormDB(count) {
    var randNum = randomInt(0, 1000);
    var res = db.postCountByTag.findOne({ random: { $gt: randNum } });
    if (res == null)
        res = db.postCountByTag.findOne({ random: { $lt: randNum } });
    return res._id;
}

function createComments(articleDate, usersCount) {
    var commentsCount = randomInt(1, 10);
    var comments = [];

    for (var j = 0; j < commentsCount; j++) {
        var year = articleDate.getFullYear();
        var month = articleDate.getMonth() + randomInt(0, 12);
        if (month > 12) {
            year++;
            month = month % 12;
        }
        var day = articleDate.getDay() + randomInt(0, 29)
        if (day > 30) {
            month++;
            day = day % 30;
        }

        var commentDate = new Date(year, month, day)
        comments[j] = {
            author: randomNickFormDB(usersCount),
            content: "boring content",
            date: commentDate,
            nestingLevel: 1
        }
    }
    return comments;
}

function createTags() {
    var tagsCount = randomInt(1, 4);
    var tags = [];
    var map = [];
    var tag;
    for (var j = 0; j < tagsCount; j++) {
        do {
            tag = randomTagFormDB(db.postCountByTag.count());
        } while (tag in map);
        map[tag] = 1;
        tags[j] = tag;
    }
    return tags;
}

function generateUsers() {
    for (var i = 0; i < 2100000; i++) {
        var n = Math.round(Math.random() * 1000);
        var m = Math.round(Math.random() * 1000);
        var k = Math.round(Math.random() * 1000);
        db.users.insert(
            {
                nick: "cooolGuy" + i,
                name: firstName[randomInt(0, firstName.length - 1)] + " " + lastName[randomInt(0, lastName.length - 1)],
                from: places[randomInt(0, places.length)],
                registerDate: randomDate(new Date(1600, 1, 1), new Date()),
            }
        )
    }
}

function generatePosts() {
    var usersCount = db.users.count();

    for (var i = 45000; i < 1000000; i++) {
        var author = randomNickFormDB(usersCount);
        var userRegisterdDate = db.users.findOne({ nick: author }).registerDate;
        var year = userRegisterdDate.getFullYear() + randomInt(0, 5);
        var month = userRegisterdDate.getMonth() + randomInt(0, 12);

        if (month > 12) {
            year++;
            month = month % 12;
        }

        var day = userRegisterdDate.getDay() + randomInt(0, 29)
        if (day > 30) {
            month++;
            day = day % 30;
        }

        var articleDate = new Date(year, month, day);

        db.posts.insert(
            {
                title: "awesome title " + i,
                author: author,
                content: "boring article",
                comments: createComments(articleDate, usersCount),
                tags: createTags(),
                date: articleDate,
                url: "boring url"
            }
        )
        usersCount++;
    }
}
\end{verbatim}
\section{CRUD}
CRUD операции осуществлялись из С\# приложения, которое использовало для этого С\# Driver.
\begin{lstlisting}
class RequestsToBase
{

    public static async Task insertUser(User user)
    {
        var blogContext = new BlogContext();
        await blogContext.Users.InsertOneAsync(user);
    }

    public static async Task<bool> tryInsertUser(User user)
    {
        var blogContext = new BlogContext();
        var filter = Builders<User>.Filter.Eq(x => x.nick, user.nick);
        var document = await blogContext.Users.Find(filter).FirstOrDefaultAsync();
        if (document == null)
        {
            await blogContext.Users.InsertOneAsync(user);
            return true;
        }
        return false;
    }

    public static async Task<bool> tryInsertPost(Post post)
    {
        var blogContext = new BlogContext();
        var builder = Builders<Post>.Filter;
        var filter = builder.Eq(x => x.title, post.title) & builder.Eq(x => x.author, post.author);
        var document = await blogContext.Posts.Find(filter).FirstOrDefaultAsync();
        if (document == null)
        {
            await blogContext.Posts.InsertOneAsync(post);
            return true;
        }
        return false;
    }

    public static async Task insertPost(Post post)
    {
        var blogContext = new BlogContext();
        await blogContext.Posts.InsertOneAsync(post);
    }

    public static async Task<List<Post>> findPost(string author)
    {
        var blogContext = new BlogContext();
        var filter = Builders<Post>.Filter.Eq("author", author);
        var posts = await blogContext.Posts.Find(filter).ToListAsync();
        return posts;
    }

    public static async Task<List<User>> findUserByNick(string nick)
    {
        var blogContext = new BlogContext();
        var filter = Builders<User>.Filter.Eq("nick", nick);
        var users = await blogContext.Users.Find(filter).ToListAsync();
        return users;
    }

    public static async Task<List<User>> findUserByName(string name)
    {
        var blogContext = new BlogContext();
        var filter = Builders<User>.Filter.Eq("name", name);
        var users = await blogContext.Users.Find(filter).ToListAsync();
        return users;
    }

    public static async Task updatePerson(SearchPost spost, Post post)
    {
        var blogContext = new BlogContext();
        var result = await blogContext.Posts.UpdateOneAsync(
            spost.ToBsonDocument(),
            post.ToBsonDocument());

        Console.WriteLine("Найдено по соответствию: {0}; обновлено: {1}",
            result.MatchedCount, result.ModifiedCount);
    }

    public static async Task DeletePerson(BsonDocument doc)
    {
        var blogContext = new BlogContext();
        var filter = doc;
        await blogContext.Posts.DeleteOneAsync(filter);
    }

    public static async Task<int> getCommentsCountByUserTag(string nick)
    {
        throw new NotImplementedException();
    }

    public static async Task<Dictionary<string, List<Post>>> getPostsbyTags(string tag)
    {
        throw new NotImplementedException();
    }

    public static async void getPostsByDate()
    {
        var blogContext = new BlogContext();
        var collection = blogContext.Users;
        var userCollection = await collection.Aggregate()
            .Match(new BsonDocument { { "nick", new BsonDocument { { "$gte", "xv" }, { "$lt", "xz" } } } })
            .Group(new BsonDocument { { "_id", new BsonDocument { { "nick", "$nick" }, { "registerDate", "$registerDate" } } } })
            .Project(new BsonDocument { { "Nick", "$_id.nick" }, { "rDate", "$_id.registerDate" } })
            .ToListAsync();
        foreach (var human in userCollection)
        {
            Console.WriteLine("Nick:\t" + human.GetValue("Nick"));
            Console.WriteLine("rDate:\t" + human.GetValue("rDate"));
            Console.WriteLine();
        }
    }
}
\end{lstlisting}
Представление документов в C\#
\begin{lstlisting}
public class Post
{
    [BsonId]
    public ObjectId Id { get; set; }
    public string title { get; set; }
    public string author { get; set; }
    public string content { get; set; }
    [BsonIgnoreIfNull]
    public IEnumerable<Comment> comments { get; set; }
    public IEnumerable<string> tags { get; set; }
    public DateTime date { get; set; }
    [BsonIgnore]
    public string url { get; set; }
}

public class User
{
    [BsonId]
    public ObjectId id { get; set; }
    [BsonRequired]
    public string nick { get; set; }
    [BsonRequired]
    public DateTime registerDate { get; set; }
    [BsonIgnoreIfNull]
    public string name { get; set; }
    [BsonIgnoreIfNull]
    public string[] from { get; set; }
}


public class Comment
{
    public string author { get; set; }
    public string content { get; set; }
    public DateTime date { get; set; }
    public int nestingLevel { get; set; }
}
\end{lstlisting}
\section{Алгоритмы поиска}
\begin{verbatim}
db.system.js.save(
   {
       _id: "levenshtein",
       value: function (s1, s2, costs) {
           var i, j, l1, l2, flip, ch, chl, ii, ii2, cost, cutHalf;
           l1 = s1.length;
           l2 = s2.length;

           costs = costs || {};
           var cr = costs.replace || 1;
           var cri = costs.replaceCase || costs.replace || 1;
           var ci = costs.insert || 1;
           var cd = costs.remove || 1;

           cutHalf = flip = Math.max(l1, l2);

           var minCost = Math.min(cd, ci, cr);
           var minD = Math.max(minCost, (l1 - l2) * cd);
           var minI = Math.max(minCost, (l2 - l1) * ci);
           var buf = new Array((cutHalf * 2) - 1);

           for (i = 0; i <= l2; ++i) {
               buf[i] = i * minD;
           }

           for (i = 0; i < l1; ++i, flip = cutHalf - flip) {
               ch = s1[i];
               chl = ch.toLowerCase();

               buf[flip] = (i + 1) * minI;

               ii = flip;
               ii2 = cutHalf - flip;

               for (j = 0; j < l2; ++j, ++ii, ++ii2) {
                   cost = (ch === s2[j] ? 0 : (chl === s2[j].toLowerCase()) ? cri : cr);
                   buf[ii + 1] = Math.min(buf[ii2 + 1] + cd, buf[ii] + ci, buf[ii2] + cost);
               }
           }
           return buf[l2 + cutHalf - flip];
       }
   }
)

db.system.js.save(
   {
       _id: "DL",
       value: function (source, target) {
           if (!source || source.length === 0)
               if (!target || target.length === 0)
                   return 0;
               else
                   return target.length;
           else if (!target)
               return source.length;
           var sourceLength = source.length;
           var targetLength = target.length;
           var score = [];
           var INF = sourceLength + targetLength;
           score[0] = [INF];
           for (var i = 0 ; i <= sourceLength ; i++) {
             score[i + 1] = []; score[i + 1][1] = i; score[i + 1][0] = INF;
           }
           for (var i = 0 ; i <= targetLength ; i++) {
             score[1][i + 1] = i; score[0][i + 1] = INF;
           }
           var sd = {};
           var combinedStrings = source + target;
           var combinedStringsLength = combinedStrings.length;
           for (var i = 0 ; i < combinedStringsLength ; i++) {
               var letter = combinedStrings[i];
               if (!sd.hasOwnProperty(letter))
                   sd[letter] = 0;
           }
           for (var i = 1 ; i <= sourceLength ; i++) {
               var DB = 0;
               for (var j = 1 ; j <= targetLength ; j++) {
                   var i1 = sd[target[j - 1]];
                   var j1 = DB;
                   if (source[i - 1] == target[j - 1]) {
                       score[i + 1][j + 1] = score[i][j];
                       DB = j;
                   }
                   else
                       score[i + 1][j + 1] = Math.min(
                           score[i][j],
                           Math.min(score[i + 1][j], score[i][j + 1])
                           ) + 1;
                   score[i + 1][j + 1] = Math.min(
                       score[i + 1][j + 1],
                       score[i1][j1] + (i - i1 - 1) + 1 + (j - j1 - 1)
                       );
               }
               sd[source[i - 1]] = i;
           }
           return score[sourceLength + 1][targetLength + 1];
       }
   }
)

db.system.js.save(
   {
       _id: "Hamming",
       value: function (strand1, strand2) {
           function shortestStrand(strand1, strand2) {
               if (strand1.length > strand2.length) {
                   return strand2;
               }
               else {
                   return strand1
               };
           };
           function strandCounting(shortest, splitStrand1, splitStrand2) {
               var count = 0
               for (var i = 0; i < shortest.length; i++) {
                   if (splitStrand1[i] != splitStrand2[i]) {
                       count += 1
                   };
               };
               return count;
           };
           var splitStrand1 = strand1.split("");
           var splitStrand2 = strand2.split("");
           var shortest = shortestStrand(splitStrand1, splitStrand2)
           var result = strandCounting(shortest, splitStrand1, splitStrand2);
           return result;
       }
   }
)
//Непосредственно сама функция поиска
function search(key, func) {
    var peopleInUsers = db.users.find({ 'name': { $exists: true } });

    var distances = [];
    while (peopleInUsers.hasNext()) {
        obj = peopleInUsers.next();
        var dist = func(obj["name"], key);
        if (dist > 0 && dist < 10) {
            distances.push({ name: obj["name"], dist: dist });
        }
    }
    distances.sort(function (a, b) { return a["dist"] - b["dist"] });
    var length;
    if (distances.length < 12) {
        length = distances.length;
    }
    length = 10;

    for (index = 0; index < length; index++) {
        print(distances[index].name);
    }

}
\end{verbatim}
\section{MapReduce}
\begin{verbatim}
function mapFunction() {
    for (var i = 0; i < this.tags.length; i++) {
        emit(this.tags[i], 1);
    }
};

function reduceFunction(key, values) {
    var sum = 0;
    for (var i in values) {
        sum += values[i];
    }
    return sum;
}

(function () {
    db.posts.mapReduce(mapFunction, reduceFunction, { out: "postCountByTag" });
})();
\end{verbatim}
\section{Создание кластера}
Настройка конфиг сервера:
Натсройка шарда №1:
Настройка шарда №2:
Натсройка шарда №3:
Настрока mongos'а:
Скрипт инициализации шардинга:

\end{document}
